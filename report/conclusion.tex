\section{Conclusion and Future Work}
\label{sec:conclusion}

We develop AccWeb, a web service that performs prefetching to improve user experience of web browsing. AccWeb has a mechanism to predict the URL the user wants to visit when he/she is typing in the browser's address bar, and prefetches the static resources associated with this URL. The user is allowed to select the types of resources to prefetch based on personal preference and network condition. Preliminary experimental results show the effectiveness of AccWeb in reducing page loading times.


There are several directions for further investigation, which we are unable to study due to time limitation.
First, it would be great if AccWeb can adjust to network condition automatically, so that it will switch to limit mode or suggest the user to do so when the network bandwidth is limited.
Second, designing an accurate prediction algorithm that works for different types of users is an interesting problem even in its own. One might borrow ideas from machine learning, and take into consideration many kinds of user behaviors in the prediction process.
Third, how to store the data (user browsing history, potential URLs to be prefetched, etc.) efficiently is a problem worth being studied.

